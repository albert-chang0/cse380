\documentclass[letterpaper,10pt]{article}

\usepackage{listings}
\usepackage{color}
\usepackage{tikz}

\setlength{\headheight}{0in}
\setlength{\marginparsep}{0in}
\setlength{\footskip}{0in}
\setlength{\headsep}{0in}
\setlength{\marginparwidth}{0in}
\setlength{\marginparpush}{0in}
\setlength{\voffset}{0in}
\setlength{\hoffset}{-1in}
\setlength{\voffset}{-1in}
\setlength{\oddsidemargin}{0.75in}
\setlength{\evensidemargin}{0.75in}
\setlength{\topmargin}{0.75in}
\setlength{\textheight}{9.5in}
\setlength{\textwidth}{7in}
\setlength{\parindent}{0in}
\setlength{\parskip}{10pt} %change this to match font size

\pagestyle{empty}
\definecolor{gray}{gray}{0.75}
\usetikzlibrary{shapes,arrows,calc}
\lstset{numbers=left,backgroundcolor=\color{gray},frame=single}

% define block styles
\tikzstyle{line} = [draw, -latex']
\tikzstyle{block} = [draw, rectangle, text centered, minimum height=2em]
\tikzstyle{mlblock} = [draw, rectangle, text width=10em, text centered, minimum height=2em]
\tikzstyle{decision} = [draw, diamond, text width=4.5em, text centered, node distance=3cm, inner sep=0pt]
\tikzstyle{cloud} = [draw, rectangle, text centered, rounded corners, minimum height=2em]

\begin{document}
    Albert Chang and Nipun Chopra\\
    CSE-380 A6\\
    University at Buffalo\\
    Dr. Kris Schindler\\
    March 1, 2011\\
    \textit{Lab 4 Documentation}

    The objective of Lab 4 was to learn about using the general purpose
    input/output, GPIO, and how to factor code using \textbf{export} and
    \textbf{extern}. There are two parts to the program.

    First the user pushes each of the four push buttons. As the buttons are
    pushed, the corresponding LEDs light up. After each individual button has
    been pushed once, the program moves onto the second part of the program.

    The second part makes use of 
\end{document}
